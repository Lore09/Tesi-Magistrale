\section*{}
\begin{center}
{\Huge \textbf{Abstract}}
\vspace{15mm}
\end{center}
\thispagestyle{empty}

Attualmente, l’orchestrazione e l’offloading dei sistemi nell’Edge-Cloud continuum si basano su approcci tradizionali che integrano tecnologie consolidate come Docker e Kubernetes con paradigmi serverless emergenti, consentendo la distribuzione dei carichi di lavoro in ambienti distribuiti. Questi metodi, sebbene efficaci in alcune applicazioni, mostrano criticità rilevanti: tempi di avvio elevati, inefficienze nell’impiego delle risorse e difficoltà nel gestire applicazioni indirizzate ad infrastrutture eterogenee. Questo limita la portabilità delle soluzioni containerizzate e le possibili implementazioni in architetture variegate presenti nel panorama dell'Edge-Cloud Continuum.\\
Per affrontare questi limiti, in questa tesi proponiamo PELATO, un framework innovativo che adotta il paradigma Function as a Service (FaaS) per permettere l’esecuzione on-demand di moduli WebAssembly. Il framework integra l’orchestrazione e la containerizzazione offerte da Docker e Kubernetes con soluzioni emergenti come wasmCloud e si avvale del modello a componenti di Wasm, aderendo alla specifica WASI per garantire interazioni sicure ed efficienti con il sistema operativo. Inoltre, PELATO implementa un sistema di networking dei messaggi basato su NATS, utilizzando il modello Pub/Sub per assicurare una comunicazione asincrona, resiliente e distribuita tra i vari componenti, facilitando così il bilanciamento del carico e riducendo la latenza negli ambienti Edge.\\
La validazione sperimentale è stata condotta attraverso quattro test progettati per misurare le performance e la scalabilità di PELATO in diverse configurazioni, variando il numero di task, le modalità di esecuzione (sequenziale o parallelizzata) e la composizione delle applicazioni. I risultati evidenziano che PELATO riduce drasticamente i tempi di esecuzione una volta generate le immagini Docker. In modalità parallelizzata, il framework ha ottenuto tempi di esecuzione tre volte inferiori rispetto a quelli sequenziali, dimostrando una gestione ottimizzata per applicazioni con più task e una scalabilità lineare con piattaforme multi-core. Inoltre, nei test di failover, il sistema ha mostrato la capacità di recupero rapido, con un tempo medio di ripristino di 23 secondi dopo il fallimento di un nodo Edge.\\
Questi risultati confermano l’efficacia di PELATO nel gestire carichi di lavoro complessi e dinamici, rendendolo una soluzione scalabile e performante per applicazioni distribuite nell’Edge-Cloud continuum.