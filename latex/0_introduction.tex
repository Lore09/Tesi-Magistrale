Negli ultimi anni il paradigma del Cloud Computing\cite{Pallis2010Cloud} ha subito una trasformazione radicale: le soluzioni tradizionali, basate su infrastrutture centralizzate, si sono evolute verso modelli distribuiti e dinamici, in grado di rispondere alle crescenti esigenze di scalabilità, flessibilità e riduzione della latenza. L’avvento della containerizzazione\cite{Felter2015An}, del serverless computing e delle architetture a microservizi ha spostato l’attenzione verso sistemi che permettono una gestione più efficiente delle risorse, riducendo il carico amministrativo e operativo. In questo contesto, WebAssembly (Wasm)\cite{WasmSpec24} si presenta come una tecnologia innovativa, capace di eseguire codice precompilato con prestazioni quasi native e offrendo una portabilità senza precedenti, indipendentemente dall’architettura hardware sottostante.\\
Il concetto di Edge-Cloud Continuum\cite{Hoshikawa2019Edge} rappresenta la convergenza tra le potenti risorse centralizzate del Cloud e i vantaggi offerti dai sistemi Edge, in cui la vicinanza ai dati permette di ottenere bassi tempi di risposta e una maggiore reattività. Tale integrazione risulta fondamentale in scenari dove la latenza e la resilienza sono requisiti imprescindibili, come ad esempio nelle applicazioni IoT\cite{Laghari2021A} e nei sistemi di monitoraggio in tempo reale. Tuttavia, l’adozione di soluzioni ibride che coniugano Cloud ed Edge solleva nuove e significative sfide, in particolare nell’orchestrazione e nella gestione dei carichi di lavoro in ambienti eterogenei. Sebbene i metodi tradizionali siano efficaci in alcune applicazioni, essi mostrano criticità rilevanti, tra cui tempi di avvio elevati, inefficienze nell’utilizzo delle risorse e difficoltà nell’adattarsi a infrastrutture variegate. Queste problematiche limitano la portabilità delle soluzioni containerizzate e ostacolano l’implementazione di applicazioni in architetture caratterizzate da una distribuzione eterogenea di risorse e tecnologie, tipica del panorama dell’Edge-Cloud Continuum. Per affrontare tali sfide, è necessaria un’innovazione che superi i limiti delle soluzioni attuali, garantendo al contempo scalabilità, flessibilità e tempi di risposta ottimizzati\cite{10682874}.\\
La presente tesi si propone di affrontare queste sfide attraverso lo sviluppo di PELATO, un framework innovativo per l’automazione e l’orchestrazione di moduli WebAssembly nell’Edge-Cloud Continuum. PELATO facilita la gestione automatizzata del lifecycle dei modulo Wasm, occupandosi della generazione e compilazione dell'applicazione e del suo deployment nell'infrastruttura. L’obiettivo principale è quello di combinare l’efficienza esecutiva e la portabilità dei moduli Wasm con la flessibilità offerta dalle tecnologie di containerizzazione e orchestrazione come Kubernetes\cite{Rejiba2022Custom}, basandosi su tecnologie emergenti come wasmCloud\cite{wasmcloud}. Attraverso questa integrazione, PELATO permette la realizzazione di applicazioni modulari, portatili e scalabili, capaci di adattarsi dinamicamente alle variazioni del carico e garantendone la distribuzione in tutto l'ambiente Edge-Cloud.\\
Il lavoro si articola in più fasi. Nel primo capitolo viene presentata una panoramica completa dello stato dell’arte, analizzando l’evoluzione delle tecnologie Cloud ed Edge, le tecniche di containerizzazione e i modelli di orchestrazione, e approfondendo il ruolo emergente di WebAssembly. Verranno discussi i vantaggi e i limiti delle soluzioni attuali, individuando le aree in cui l’integrazione tra Wasm e strumenti di orchestrazione può apportare significativi miglioramenti in termini di performance, flessibilità e gestione delle risorse.\\
Nel secondo capitolo viene illustrata l’architettura del framework proposto, con particolare attenzione alla modularità e all’interoperabilità dei suoi componenti. Verranno descritti il design del sistema, le scelte tecnologiche adottate e le strategie implementate per garantire sicurezza e resilienza in ambienti distribuiti. Particolare enfasi sarà posta sull’utilizzo di interfacce standardizzate, in grado di facilitare l’integrazione di moduli sviluppati in linguaggi diversi e di supportare un approccio orientato ai microservizi.\\
I capitoli successivi approfondiscono in dettaglio i processi di generazione, build e deployment del framework, evidenziandone l’architettura e le modalità operative. Nel capitolo dedicato alla generazione, vengono descritti gli strumenti e le tecniche utilizzate per configurare e definire workflow modulari, con particolare attenzione alla creazione automatizzata di task e alla compilazione dei template Wasm. Nel capitolo sulla build, viene esaminato il processo di compilazione e packaging dei moduli WebAssembly, analizzando le ottimizzazioni introdotte, come la gestione delle immagini Docker\cite{Anderson2015Docker} in cache e l’esecuzione parallelizzata per ridurre i tempi complessivi. Infine, nel capitolo sul deployment, si analizzano le strategie adottate per distribuire le applicazioni in ambienti eterogenei, sfruttando strumenti come wasmCloud per il bilanciamento del carico e la resilienza, e mostrando come PELATO gestisce dinamicamente il failover e l’allocazione dei task.\\
Questi capitoli sono arricchiti da flussi di lavoro automatizzati e pipeline di integrazione continua, descritti con esempi pratici e casi di studio che dimostrano l’efficacia del framework in scenari operativi reali, sia in ambienti Cloud che Edge.\\
Infine, il lavoro si conclude con una valutazione delle performance ottenute, evidenziando il contributo del framework allo sviluppo di soluzioni applicative più efficienti e resilienti in un contesto distribuito. Saranno proposte possibili direzioni future di ricerca, finalizzate all’ulteriore ottimizzazione dell’integrazione tra tecnologie Cloud ed Edge e all’ampliamento delle funzionalità fornite dai componenti WebAssembly\cite{webassembly_component_model}.\\
In un’epoca in cui la rapidità, la flessibilità e la capacità di adattarsi a contesti dinamici sono requisiti essenziali per le applicazioni moderne, questa tesi si propone come un contributo significativo al progresso tecnologico, offrendo un approccio innovativo all’orchestrazione e automazione dei moduli Wasm. Il framework sviluppato rappresenta un passo avanti verso la realizzazione di sistemi distribuiti capaci di sfruttare appieno le potenzialità dell’Edge-Cloud Continuum, promuovendo una nuova generazione di applicazioni scalabili e performanti.