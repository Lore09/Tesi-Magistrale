Questo lavoro ha portato allo sviluppo di PELATO, un framework che rappresenta una soluzione innovativa per l’automazione e l’orchestrazione di moduli WebAssembly, adottando il paradigma Function as a Service (FaaS). PELATO consente di trasformare il codice in unità modulari ed eseguibili in maniera dinamica nell’Edge-Cloud Continuum, ottimizzando l’utilizzo delle risorse e migliorando la scalabilità in ambienti eterogenei. Attraverso l’adozione di tecnologie come Docker, Kubernetes e wasmCloud, PELATO combina l’efficienza dell’esecuzione quasi nativa dei moduli Wasm con la flessibilità necessaria per gestire carichi di lavoro complessi e dinamici.\\
Il framework è stato sviluppato seguendo un approccio strutturato che copre tre fasi principali: generazione, build e deployment. Nel capitolo dedicato alla generazione, viene affrontato il processo di configurazione e definizione dei workflow, con un focus sulla creazione automatizzata di task e sulla compilazione dei template Wasm. Questo consente agli sviluppatori di generare moduli Wasm in modo standardizzato e ottimizzato. Il capitolo dedicato alla build analizza il processo di compilazione dei moduli, implementando tecniche per ridurre i tempi di esecuzione, come la gestione delle immagini Docker in cache e l’esecuzione parallelizzata delle operazioni. Infine, il capitolo sul deployment esplora le strategie adottate per distribuire i moduli in ambienti Cloud ed Edge, sfruttando il modello a componenti di Wasm e il sistema di networking basato su NATS, che adotta il paradigma Pub/Sub per garantire comunicazioni affidabili e asincrone tra i componenti del sistema.\\
I risultati sperimentali evidenziano che PELATO è in grado di generare, buildare e deployare moduli Wasm garantendo prestazioni competitive in termini di tempi di avvio e overhead computazionale rispetto alle soluzioni containerizzate classiche. La validazione sperimentale è stata condotta attraverso quattro test principali, progettati per misurare le performance e la scalabilità del framework in diverse configurazioni operative, variando il numero di task, le modalità di esecuzione (sequenziale o parallelizzata) e la composizione delle applicazioni.\\
In modalità parallelizzata, PELATO ha mostrato tempi di esecuzione tre volte inferiori rispetto a quelli sequenziali, dimostrando una gestione ottimizzata per applicazioni con più task e una scalabilità lineare su piattaforme multi-core. Inoltre, la gestione delle immagini Docker in cache ha ridotto significativamente i tempi di avvio della fase di Build, migliorando l’efficienza complessiva del processo. Nei test di failover, il framework basato sull'infrastruttura wasmCloud ha evidenziato una capacità di recupero rapido, con un tempo medio di ripristino di 23±1,5 secondi dopo il fallimento di un nodo Edge. Questo approccio consente una gestione ottimizzata delle risorse, riducendo latenza e overhead, e garantisce una risposta tempestiva alle variazioni del carico di lavoro.\\
Integrandosi armoniosamente con infrastrutture moderne, PELATO facilita il deployment e la manutenzione delle applicazioni distribuite, eliminando la necessità di interventi manuali complessi. I tempi completi dell’esecuzione della pipeline con 20 task configurate si aggirano sui 175 secondi, un risultato in linea con i tempi di Build delle applicazioni containerizzate standard, confermando la competitività del framework rispetto alle soluzioni tradizionali.\\
Tra gli sviluppi futuri, si prevede l’implementazione di nuovi template che possano ampliare ulteriormente le funzionalità del framework. Questi template saranno progettati per semplificare la configurazione dei workflow e per favorire una più stretta integrazione con altri sistemi e tecnologie emergenti, come soluzioni di intelligenza artificiale, strumenti di monitoraggio avanzato e sistemi di persistenza dei dati. Tale evoluzione renderà il framework ancora più versatile e capace di adattarsi a una gamma più ampia di scenari applicativi, sfruttando appieno le potenzialità offerte dalla convergenza tra Cloud ed Edge.\\
Un ulteriore miglioramento potrebbe consistere nell’efficientamento delle procedure di build mediante l’adozione di container più minimali, in grado di ridurre ulteriormente l’overhead e migliorare i tempi di deploy dei moduli Wasm. Parallelamente, estendere il supporto a linguaggi diversi da Go rappresenta una prospettiva interessante: integrare toolchain per linguaggi come Rust, C++ e altri emergenti arricchirebbe la versatilità del framework, offrendo agli sviluppatori una gamma più ampia di opzioni per la generazione dei moduli e favorendo l’adozione della soluzione in contesti applicativi sempre più diversificati.\\
In conclusione, il framework PELATO rappresenta un contributo significativo all’orchestrazione dei moduli WebAssembly, offrendo una soluzione efficiente, resiliente e facilmente scalabile per la gestione delle applicazioni distribuite. I risultati ottenuti e le prospettive di sviluppo indicano un percorso di evoluzione continuo, che potrà ulteriormente potenziare l’efficacia della piattaforma e aprire nuove opportunità nel panorama del computing distribuito.